\section{Introduction}
\label{sec:intro}

% phishing is important
Web-based phishing attacks, where a webpage, through mimicking or urgency, tricks the user into submitting personal information to an attacker, have been increasing for the last 5 years~\cite{apwg}. Phishing attacks can have high-profile targets, like the NGOs and government workers targeted in 2021~\cite{intelligenceNewSophisticatedEmailbased2021,SophisticatedSpearphishingCampaign} and lead to more sophisticated cyberattacks and data-breaches~\cite{ThreatActorLeverages2024}. Phishing kits, ready-to-deploy software packages sold at illicit markets for launching phishing attacks, have lowered the barrier of entry for malicious actors. Sellers often market phishing kits as bundles of quality-of-life features for attackers, such as built-in evasions from automated crawlers, exfiltration to Telegram channels, and obfuscation\cite{tejaswiLeakyKitsIncreased2022,oest_inside_2018}. Deploying these kits can be as easy as uploading them to free hosting providers and mass sending multiple links that exfiltrate the credentials to an endpoint you control.

% JavaScript is important because of evasions
One of the selling points of phishing kits is evasion from researches and analysts, extending a phishing page's lifetime (delta time between deployment and discovery), and increasing the number of victims visiting the page without a browser warning. While server-side logic of kits can only make assumptions about the system based on the IP address and User agent, through API calls to the browser, client-side JavaScript code can query the user's system for CPU core counts, memory overhead, request user interaction, or call out to a 3rd party bot detection like CloudFlare\footnote{Similar to the phishing page that stole credentials from Troy Hunt, the maintainer of HaveIBeenPwned\cite{SneakyPhishJust2025}}~\cite{zhangCrawlPhishLargescaleAnalysis2021,zhangImSPARTACUSNo2022}. In the end, the JavaScript logic in a kit can range from a simple user-agent-based redirection to an AES-encrypted script that dynamically decrypts itself, identifies the browser through a series of API calls, and renders the page after confirming the victim is using a mobile device. Due to their mass-deployed and sold nature, phishing kits' exfiltration point, the targeted brand, or IP blocklists divide phishing kits into phishing kit families. 

% Our paper
This paper aims to aid researchers and analysts by automatically differentiating groups of phishing pages based on the underlying kit. This automates a previously manual process and enables us to measure the popularity of different client-side techniques across these groups.
We focus on the following 10 techniques employed by phishing pages: fingerprint extraction, client-side IP check, timing-based bot detection, encoding-based obfuscation, dynamic script execution, basic fingerprint, dynamic script injection, Cloudflare turnstiles, and pop-ups. These techniques harvest credentials, evade crawlers, or obfuscate code to avoid analysis, all of which are sought-after capabilities of phishing kits. 
Leveraging prior work into client-side cloaking and web privacy, we build a mapping of different evasion techniques to browser APIs executed and provide an up-to-date and mass-deployment-resilient description of how widespread these are.
By clustering pages based on their browser API usage, we construct groups of pages based on a shared set of techniques. Evaluating these groups over a ground truth dataset of URLs to kit-family mappings, we find that clustering over the set of browser APIs executed yields an FMI-based accuracy of \gtFMI{} and a validity score of \gtVS{}. We then turn the clustering approach to \totalPagesWithJavascriptFP{} pages, which we group into \totalClusters{} clusters. Figure~\ref{fig:pages_clusters_per_month} shows how the volume of pages/week can be by an order of magnitude when viewed as active clusters/week. We then use a predefined mapping of browser APIs to phishing techniques to better understand how widespread these practices are, while controlling for the mass deployment of a single kit. 

This work stands apart from prior research, as we automatically differentiate pages using features from dynamic instead of static analysis. As these features are tied to capabilities a prospective kit buyer is looking for, they are less likely to vary between different deployments of the same kit family. Compared to the prior attempt at kit identification\cite{castanoPhiKitAPhishingKit2023}, of $F_1=9.03\%$ and $F_1=31.11\%$ with DOM clustering and URL path-based signatures, respectively, we achieve a much higher FMI of \gtFMI{} (69\% with a rebalanced ground truth). 
Overall, we offer the following contributions:

\begin{contributions}
    \item We find that \emph{browser API usage alone} is sufficient to isolate and distinguish known phishing kit families. With a ground truth dataset of \totalNumberOfKitFamilies{} kit families deployed on \totalKitPages{} URLs, we achieve accuracy metrics of a \gtFMI{} Fowlkes-Mallows score (69\% when rebalanced for a single-instance of a mass-deployed kit) and an \gtVS{} V-measure in the clusters of these pages relative to their kits used.
    \item We experimentally show that browser APIs common on the web (DOM APIs, property reads, etc.) serve as a valuable identifier for identifying kits, as they signal the kit developer's choices, and that the more sophisticated a page's client-side logic is, the more indicative it is of the underlying kit.
    \item We isolate \totalPagesClusterable{} phishing pages (out of 1.3M) with enough sophistication to sort into into \totalClusters{} clusters. By propagating techniques from member pages to the cluster as a whole, we find that UI-interactivity and basic fingerprinting are near-universal in the ecosystem. At the same time, mouse detection via browser APIs, Cloudflare Turnstile embedding, and dynamic script creation are still relatively rare. While we observe client-Side IP checks occur in \clientcheckPages{} pages they spanned across \clientcheckCluster{} clusters, and \clientcheckTopClusterPrecent{} of the pages come from a single cluster, which consists of Facebook business account phishing pages. We also find that compared to prior work, there is a decreased ratio of pages that employ pop-ups as a form of bot detection. 
    \item We release a dataset of \totalKitPages{} phishing pages labeled with their underlying kit and an unlabeled dataset of browser APIs executed on over 1.3 million.
\end{contributions}

% How are we different from prior work

\begin{figure}[ht]
  \includegraphics[width=\columnwidth]{assets/pages_clusters_per_month.png}
  \caption{Comparison between monthly pages observed vs the monthly clusters, based on dynamic behaviors, observed.}
  \label{fig:pages_clusters_per_month}
\end{figure}
