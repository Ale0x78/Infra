\section{Introduction (1 page)}
\label{sec:intro}

Web-based phishing attacks, where a web page, through mimicking or urgency, tricks the user into submitting personal information to an attack, have been increasing for the last 5 years\cite{apwg}. While varying in delivery vectors, sent through email, SMS, or QR codes, they remain a successful attack for compromising individual accounts and enterprises. In 2021, the Cybersecurity and Infrastructure Security Agency (CISA) warned of phishing campaigns targeting NGOs and government workers which smuggled the HTML and Javascript of the phishing page through an attachment\cite{intelligenceNewSophisticatedEmailbased2021,SophisticatedSpearphishingCampaign}. Once an actor steals credentials, they sell them on illicit markets or leverage them to leak more data from the target, for example in 2024, joined with the Multi-State Information Sharing and Analysis Center, CISA reported that compromised former employee credentials were being used to access internal networks within the US government\cite{ThreatActorLeverages2024}.

Client-side javascript is a keystone in this profitable and effective family of attacks. This highly dynamic language, which has access to privileged functionality through browser APIs, enabled the attacks to obfuscate the true purpose of the web page (HTML/CSS and the javascript code itself) and effectively identify settings when a victim or a potential analysis framework is viewing the web page \cite{zhangCrawlPhishLargescaleAnalysis2021,zhangImSPARTACUSNo2022}. The javascript logic can range from a simple user-agent-based redirection to an AES-encrypted script that dynamically decrypts itself, identifying the browser through a series of API calls extracting information about the host's browser capabilities. The ability to distill information about the user via the browser's capabilities is referred to as browser fingerprinting. Prior work has studied it in both phishing and web-based advertising settings. 

The Anti-Phishing Work Group (APWG) has reported over 1 million phishing attacks yearly since 2020. With the ecosystem full of pages, with Javascript reaching browser APIs to mimic the target page, cloak away from analysts, and exfiltrate data, we propose using browser API traces from these pages to study them at a layer of abstraction. Figure-\ref{fig:pages_per_month} shows the severy of monthly traffic we observed on these feeds which match the client-side javascript usage required by this paper.
We have crawled five phishing feeds for \daysCrawled{} days to answer the following research questions.
\begin{questions}
    \item \textbf{How indicative is a page's browser API trace of the underlying phishing kit? What differences do we see between the kits?} Having collected 4,448 pages from 526 distinct kits, we find that clustering based on the distance of browser APIs used is enough to identify if two pages share a kit behind the scenes, achieving an accuracy of 92\% based on Fowlkes\-Mallows index.  \label{itm:Q1}
    \item \textbf{What does the clustering of all phishing pages collected look like?} \label{itm:Q2}
    By computing the density based cluster of these pages for a 4-week window, rolling it by 2 weeks, and merging these clusters by threshold of common pages, we get 27,833 total cluster for the span of \daysCrawled{} days. These clusters are very well formed with a silhouette score of 0.83 ($\sigma=0.04$).  

    % \item \textbf{Over time, how does the usage of APIs change in phishing pages? Do phishing pages adopt emerging and experimental browser APIs?} We find experimental browser APIs (not implemented by all major browsers) present in no more than 2\% of the pages we've seen. These APIs are used for both cloaking (bot-detection and fingerprinting) and mimicking (accurately copying the victim's battery percentage). We find that 405 browser APIs exhibit change patterns in 46\% of these APIs, showing a decrease, and 43\% being a spike in daily traffic observed, signaling that adoption of new APIs is far less common by the ecosystem as a whole. \label{itm:Q3}
    
    \item \textbf{How are browser APIs being used in phishing? How are different techniques spread across clusters? } We find that 1.2 million pages (83\%) of pages we observe execute JavaScript in a first-party context. We find no more then 15\% use an obfuscation related browser API. We check API usage of techniques established by prior work and find a decrease in user-interactive permission pop-ups. We do see that 1 in 3 pages calling out to a basic fingerprinting property (userAgent, cookie, or HTTP referrer). We observer that 1 in 4 pages makes a network request upon loading, some reaching out to IP identification URLs which allow free look-ups for the requesting machine, in this case that being the victims browser. \label{itm:Q4}
\end{questions}
\begin{figure}[ht]
    \includegraphics[width=\columnwidth]{assets/pages_per_month.png}
    \caption{Total number of pages observed every month by us, that met the clustering requirements outlines in Section~\ref{sec:methods}}
    \label{fig:pages_per_month}
\end{figure}
The rest of the paper will breakdown the necessary background regarding the phishing ecosystem and adversarial techniques, we will go over relevant work in studying the ecosystem in Section-\ref{sec:related}. We will describe the methods we used to study the ecosystem in Section-\ref{sec:methods} and go over what we found in Section-\ref{sec:results}.
    