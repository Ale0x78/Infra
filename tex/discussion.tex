\section{Discussion}
\label{sec:discussion}
The complexity of a phishing page's client-side code helps analyze the massive volume of phishing pages in the wild caused by the proliferation of phishing kits. This section discusses what makes kits identifiable, why, and what the clusters represent.

\subsection{What makes a kit identifiable?}
The more sophisticated the phishing kit becomes, the easier it is to spot by the browser APIs it uses. Everything from exfiltrating a browser fingerprint to sophisticated evasions (e.g., Canvas captchas) makes mass deployments of the kit stand. There is also an economic incentive to sophistication, as novel use of browser APIs leads to better evasions from detectors\cite{zhangCrawlPhishLargescaleAnalysis2021, oestPhishFarmScalableFramework2019}, more valuable credentials harvested\cite{sanchez-rolaRodsLaserBeams2023, linPhishSheepClothing2022}, or resilience to analysis\cite{fv8-sec24,jsobf-imc20}. For example, the APIs \textit{Keyboard.lock}, \textit{HTMLDocument.onkeydown} for keyboard locking, \textit{Window.atob} for obfuscation, and a handful of fingerprint APIs and DOM APIs for dynamic content generation set the cluster shown in Figure~\ref{fig:ms_defender} apart from other pages. 

\subsection{Advantages to behavioral aggregation}
Phishing feeds are a noisy data source for studying the ecosystem, from e-commerce pages to mass-spammed USPS and EZ-parking phishing pages. As client-side code for phishing pages grows to be more complex, behavioral aggregation, akin to what has been done by prior work to identify exploit kits\cite{Kizzle}, is necessary. The methodology in this paper is aimed at researchers and analysts. For research, identifying shared kits in a dataset of phishing pages helps control for easily obtainable or mass-deployed phishing kits, measuring the prevalence of different techniques across kits, rather than pages. \emph{At worst, we overestimate how popular different techniques are across kits, meaning we provide an upper bound for the less popular techniques.} For analysts, our methodology acts as a quick way to aggregate and share phishing kits-related threat intelligence between pages. Things like server-side cloaking technique, preferred exfiltration method, ties to APTs, and data exfiltrated. While kit-families can vary in which IPs they denylist, and what user-agents they allow, fingerprinting the underlying kit can allow analysts to deduce if a page employs these techniques in the first place. 