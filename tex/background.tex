\section{Background}
\label{sec:background}

In this section, we provide a background on current developments in phishing as a phenomenon and adversarial JavaScript techniques, as well as an overview of the parties involved in the phishing ecosystem.
\subsection{Phishing as a phenomenon}
Phishing is a form of social engineering where an adversary pretends to be a trusted entity to steal a user's credentials or gain access to a specific machine, network, or account to which the user has access. While delivery mechanisms vary, the most common way of phishing inevitably leads to a web page that requests some personal information (usually credentials) from the user. 

While a basic phishing web page is relatively simple to deploy and send through social media or email, the ecosystem has proven to be an ever-changing landscape. Workgroups tracking phishing saw an uptick in phishing domains in the 2020s, with a shift in what sectors are targeted throughout every year. In Q3 of 2020, the Anti-Phishing Working Group (APWG) reported Software-as-a-Service (SAAS) and webmail services as the most targeted sectors, with 31\% of all phishing attacks targeting them. In Q3 of 2021, it was the financial sector; in 2024, it is Social media~\cite{APWGPhishingActivity}. In the 3rd quarter of 2024, the APWG reported 900,000 phishing attacks. Besides detection, to be able to identify trends in the ecosystem, security researchers need to be able to group a large number of pages through a higher level of abstraction.

As more organizations and academics turn to studying phishing websites and developing methods of identifying them, anti-analysis and anti-bot detection techniques have emerged in the ecosystem. These techniques are no different from anti-debugging techniques found in malware; the end goal is to make detection and analysis of the page difficult. 
The techniques can be broadly classified into server-side and client-side techniques~\cite{oest_inside_2018}.
Server-side techniques are stealthier; however, they rely on limited information and can be studied by acquiring a zip bundle to set up the page (sometimes called a phishing kit). Most server-side techniques rely on precompiled deny-lists or allow-lists of IP addresses, user agents, or referrers (the page from which the link is visited as identified by an HTTP header). The cloaking outcome can also vary; the page can redirect to a benign page, a long-dead phishing page, or send you off to an affiliate marketing page. Some pages may be simple checkpoints and choose to forward you to the actual phishing page if your browser does not meet the criteria the attacker requires (e.g., crawling bot, not a mobile user, not in a specific country). 

\begin{figure}[t]
  \includegraphics[width=\columnwidth]{assets/cloudflare.png}
  \caption{Example of a phishing page in our dataset that embedded Cloudflare Turnstile verification on a non-cloudflare domain}
  \label{fig:cloudflare}
\end{figure}

Client-side techniques, while allowing for much richer evasive behavior, are more detectable via instrumented tools. With the rise of fingerprinting on the web, driven by tracking scripts of advertisers and analytics companies, phishing pages turned to using browser fingerprinting APIs to identify real users and exfiltrating the fingerprinting to associate with the credentials harvested~\cite{sanchez-rolaRodsLaserBeams2023}. Phishing pages utilize Browser APIs to trigger permission pop-ups to identify headless browsers (without a user interface, usually used for crawling) and automated crawlers that can not interact with the full browser UI. Prior work has also identified CAPTCHAs and click-through pages as a form of client-side cloaking, which requires more sophisticated crawlers to combat. Cloaking also inherits behaviors from bot-detection and abuse prevention, recently, phishing pages have opted to use Cloudflare Turnstile, a popular widget for obuse prevention, in order to offload the engineering work required to accuratly identify automated browsers. Figure~\ref{fig:cloudflare} shows an exmaple of a non-cloudflare domain, using a Cloudflare Turnstile as seen from the viewpoint of our automated crawler.

Even URL features of a phishing link contain techniques that have evolved to respond to anti-phishing research. Phishing pages frequently use URL shorteners (public or private) to obfuscate the final destination, landing pages requiring a user to follow hyperlinks to the actual page, and free web hosting with trustworthy top level domains (TLDs). 

\subsection{Phishing as an ecosystem}
As an ecosystem, phishing attacks have two major branches: cybercrime as a service and credential sales. While you could develop a phishing page, set up hosting for it, adapt the latest cloaking techniques to your page, deliver the page through SMS, Email, or social media, and leverage the accounts for the financial game, the ecosystem as a whole has compartmentalized these into different segments\todocite{Citation needed}. Phishing kits (bundles of software providing ready-to-deploy phishing pages) rose in popularity in illicit markets, varying in value, sophistication, and features. One stage higher from just selling kits is Phishing-as-a-service providers, which will do the majority of the technical challenges. Both of these enable massive phishing attacks with lower technical barriers to entry and have been researched. Prior work has shown that phishing kits are not above stealing credentials from their customer's kit deployments ~\cite{freephish,mccalleyAnalysisBackDooredPhishing2011}, adapting and borrowing features from other kits~\cite{intelligenceFrankenphishTodayZooBuilt2021}, and that they are sometimes are tied to specific actors ~\cite{unit42ThreatActorGroups2024}.

The credential sales part of the ecosystem has also adapted to modern MFA/2FA practices. With prior work showing that a browser fingerprint is sometimes enough to trick their protections~\cite{lin_phish_2022}, an actual cost is associated with executing JavaScript on the phishing page~\cite{rola_rods_2023}, as accounts with a browser fingerprint go for a higher value on elicit markets. While phishing pages can risk storing the credentials on servers that can be taken down, some have shown that they should be sent to chat channels for ingestion.


\subsection{Adversarial JavaScript}
The dynamic nature of JavaScript enables a variety of techniques for concealing from analysis and detection. JavaScript obfuscation can transform a known malicious sample into an undetectable one. Webpack enables bundling benign and malicious scripts and wrangling them to make static analysis harder. The other side of obfuscation is evasions; in addition to making code comprehension (via human or machine) harder, malicious actors have deployed time bombs, offloading parts of the malicious script to be read from the DOM or via a network request, and dynamic code generation to evade revealing the entire malicious behavior and thus detection.~\cite{fv8-sec24}

The research community has tried tackling these techniques by developing deobfuscators, employing machine learning techniques from static features to identify similarities between known malicious and unknown samples, and developing force-execution engines, executing parts of the code that may trigger the malicious behavior or the next stage of an evasion.