\section{Background (1.5-2 pages)}
\label{sec:background}
\subsection{Phishing as a phenomenon}
Phishing is a form of social engineering where an adversary pretends to be a trusted entity to steal a user's credentials or gain access to a machine/network/account that the user has access to.

Workgroups tracking phishing saw an uptick in phishing domains in 202X; with a shift in what sectors are targeted.
\DraftToDo{Pull latest trends according to APWG}

As more organizations focus on phishing websites, anti-bot detection techniques have emerged in phishing pages. These techniques can be broadly classified into server-side and client-side techniques.
Server-side techniques are stealthier; however, they rely on limited information.
\DraftToDo{Summary of Oest's inside phishers mind, and the no-free-phish papers}
Client-side techniques, while allowing for much richer evasive behavior, are more detectable via instrumented tools.
\DraftToDo{List client-side techniques, who they have been studied (forced execution), how they can be studied (execution traces)}
\subsection{Phishing as an ecosystem}
Cybercrime, as a system, introduced phishing kits into the mix. These kits are ready to make phishing pages (or deployment scripts) sold in black markets. 
\DraftToDo{Write about what kits are, how prior work has studied them}
A real cost is associated with executing javascript on the phishing page, as accounts with a browser fingerprint go for a higher value on elicit markets. 
\DraftToDo{Write up Sheep-In-World's findings and RwLB's findings}
Even URL features of a phishing link contain techniques that have evolved to respond to anti-phishing research. 
\DraftToDo{URL shorteners (cite us), landing pages (cite Netcraft), abuse of free web hosting (cite the free web hosting paper)}
\subsection{Adversarial JavaScript}
Phishing pages are not the only places where JavaScript may behave adversarially. 

Obfuscation renders static analysis hard.

Prior work in studying browser fingerprinting invalid creative usage of browser APIs to identify specific users. 

Prior work in Malware and Exploit Kit detection has looked at linking them together via static features and machine learning.

The dynamic nature of JavaScript leads to many creative ways to hide.
\DraftNote{This is my attempt to have a little thread to link back to Eval + editing script tags analysis we do}