
\section{Results (3-4 pages)} 
\label{sec:results}


In this section, we will evaluate our clustering approach against the ground truth, inspect the distribution of ground truth pages in the final clusters, and report temporal patterns and the commonality of phishing behaviors across those clusters. In total, we crawled for \daysCrawled{} days collecting browser traces from \totalPageWithJavascript{} pages, out of which only \totalPagesWithJavascriptFP{} qualified for clustering by executing at least 8 APIs and \totalPagesClusterable{} pages were not sorted by HDBSCAN into a cluster with no common APIs or placed as noise.


\subsection{Cluster evaluation}

\begin{figure}[t]
  \includegraphics[width=\columnwidth]{assets/intra_kit_sim.png}
  \caption{Histogram of the Jaccard index based similarity between any pair of pages belonging to the same kit, versus from different kits}
  \label{fig:intra_kit_sim}
\end{figure}

\paperFinding{Pages from different kits employ vastly different APIs} Figure-\ref{fig:intra_kit_sim} shows the distribution of Jecard-based similarity between pages from the same kit versus different kits. We observe that while most pages with the same kit, as identified by KitPhisher, have around 90-100\% similarity, pages from different kits, even ones with similar themes, rarely exceed 60\% similarity. We leverage this in our clustering by treating $1 - \text{JI}(A,B)$ as a distance kernel of two pages. 

\paperFinding{Browser API usage unique identify phishing kits apart from eachother} Clustering pages from \totalKitPages{} pages across \totalKits{} kits, yeild \totalKitClusters{} clusters. Evaluating these clusters against the ground truth labels for each page, we find that our clusters have an FMI-based accuracy of \gtFMI{}. The clusters have a V-Score of \gtVS{}, which increases with a higher $\beta$, meaning the clusters tend to combine two kits into one instead of splitting pages from the same kit across multiple clusters.

% Ignoring some types of APIs, reduces the number of pages you can cluster, while not impacting the FMI or V-Score
\paperFinding{DOM APIs and property accesses play an essential role in identifying the underlying kit} With \TotalAPIs{} browser APIs, feature reduction becomes an obvious goal. However, removing DOM-related APIs, or property reads, out of consideration drastically reduces the number of pages we can consider for ground truth evaluation, not increasing our overall accuracy. Evaluated our same methodology described in Section-\ref{sec:methods} with all HTML-DOM, SVG, and CSS APIs removed and observed FMI-based accuracy of 0.93 and V-Score of 0.86. When all property reads were removed (often used in finterprinting\todocite{Junhua}), we saw FMI and V-Score of 0.93 and 0.85, respectively. In both cases, we can cluster fewer pages and thus identify fewer kits. Finally, to ensure that the distance metric is sufficient to isolate pages from different kits in most cases, we attempt to cluster all XXX pages from YYY kits, even if we only observed a single page from that kit. Allowing HDBSCAN to form singleton clusters, we see that these pages can still be differentiated from one another with an FMI of XXXX and V-Score of YYYY.


% We will discuss this later in the discussion, but it should be noted that we aggregate pages across different languages and deployment infra
\begin{figure*}[t]
  \centering
  \subfloat[\centering Page variant in English]{{\includegraphics[width=3in]{assets/ms_defender_en.jpg} }}%
  \qquad
  \subfloat[\centering Page variant in Japanese]{{\includegraphics[width=3in]{assets/ms_defender_jp.jpg} }}%
  \caption{Cropped screenshots from Cluster-53d5c420, IP addresses and location redacted to ensure anonymity of the authors.}%
  \label{fig:ms_defender}%
\end{figure*}

\paperFinding{Clustering large quantities of phishing pages, based on their browser APIs, keeps pages deployed via the same kit together}
Pages with ground truth labels for originating kits remain appropriately sorted out in the clustering of the \totalPagesClusterable{} pages, as out of the 3,505 pages clustered within the larger clusters, we maintain an FMI of 0.948 and a V-Score of 0.889, respectively.
Manually inspecting the clusters, we observe that these clusters unique pages across deployment types (AWS, Cloudflare, DigitalOcean, etc.) and languages. For example, Cluster-e325887b comprises 487 pages across 5 unique tldr and contains pages engaging in voice-based phishing attacks (tech support scams) across Japanese, English, and German, varying phone numbers and errors in each, shown in Figure-\ref{fig:example}. With over 400 APIs in common, it is clear that these pages' usage of keyboard intercepting APIs, Audio APIs, and Network APIs for IP intelligence caused the cluster to be formed.

\subsection{Temporal patterns of clusters}
With \totalClusters{} clusters, we observe 4 temporal patterns: constant, abrupt, seasonal, and re-emergant. 

\paperFinding{Majority of phishing  pages, originate from the same 50 clusters}
\totalFromTop{} pages of pages that exectued javascript in a first party context (\totalPagesWithJavascriptFP{}) are sorted into one of the 50 top clusters.

\subsection{Phishing Techniques across clusters}

\section{Discussion (1 page)}
\label{sec:discussion}
\subsection{Why clusters?}
\subsection{Dynamic analysis combat evasions}
% \subsubsection{}