\begin{abstract}
Phishing remains one of the greatest security challenges facing users, while the Anti-Phishing Work Group reported over one million phishing pages in the last year alone.
Phishers achieve this scale by using phishing kits --- ready-to-deploy phishing websites --- to rapidly deploy phishing campaigns with similar data exfiltration, evasion, or mimicry techniques.
In contrast, researchers and defenders continue to fight phishing on a page-by-page basis, and they still rely on manual analysis to recognize static features for kit identification. 
  
This paper aims to aid researchers and analysts in 
studying phishing by automatically clustering pages in large datasets via the page's browser API usage.
In most cases, our techniques separate pages by the underlying phishing kits from which they originate.
Our system has an accuracy of 93\% on a ground-truth dataset of pages and kits collected from the wild. With a curated mapping of techniques to browser APIs and over 500,000 pages in which we identify 11,377 clusters, we explore what techniques are universal, widespread across kits, or kit-specific.
We find UI interactivity and basic fingerprinting (User-agent, Cookies, and referer) to be the most universal techniques, present in 90\% and 80\% of the clusters, respectively. 
On the other hand, mouse detection via the browser's mouse API is among the rarest behaviors, despite being used in a deployment of a 7-year-old open-source phishing kit. 
Overall, we show how defenders can leverage the increasing JavaScript complexity of phishing pages against adversaries, identifying phishing pages originating from the same kits, thus enabling defenders to work at more scalable abstraction levels.
\end{abstract}