\begin{abstract}
Phishing remains one of the most significant security challenges facing users, meanwhile the Anti-Phishing Work Group reported over one million phishing pages in the last year alone.
Phishers achieve this scale by using phishing kits --- ready-to-deploy phishing websites --- to rapidly deploy phishing campaigns with marketable data exfiltration, evasion, or mimicry techniques.
In contrast, researchers and defenders continue to fight phishing on a page-by-page basis, and they still rely on manual analysis to recognize static features for kit identification. 
  
This paper aims to aid researchers and analysts by automatically differentiating groups of phishing pages based on the underlying kit, automating a previously manual process, and enabling the measurement of techniques per kit and how widespread some kits are. Our system has an accuracy of 95\% on a ground-truth dataset of \totalNumberOfKitFamilies{} kits collected from \totalKitPages{} pages.
On an unlabeled dataset, we leverage the complexity of \totalPagesClusterable{} phishing pages' JavaScript logic to group them into \totalClusters{} clusters, annotating the clusters with what phishing techniques they employ and exploring which techniques are universal, widespread across kits, or kit-specific. We find that UI interactivity and basic fingerprinting (User-agent, Cookies, and Referrer) are the most universal techniques, present in 90\% and 80\% of the clusters, respectively. On the other hand, mouse detection via the browser's mouse API is among the rarest behaviors, despite being used in a deployment of a 7-year-old open-source phishing kit.
Overall, we show how defenders can leverage the increasing JavaScript complexity of phishing pages against adversaries, identifying phishing pages originating from the same kits, thus enabling defenders to work at more scalable abstraction levels.
\end{abstract}