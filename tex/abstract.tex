\begin{abstract}
  Phishing kits, ready-to-deploy software packages for phishing websites, have proliferated numerous web phishing attacks launched daily. Last year alone, the Anti-Phishing Work Group reported over one million phishing pages. This volume presents new challenges for security researchers and analysts, as large quantities of pages employing behaviors of interest (data exfiltration, evasion, or mimicking techniques) can originate from a singular phishing kit while being deployed across different infrastructures throughout the year. At the same time, the whole ecosystem is examined page by page, and manual analysis is required to recognize static features for kit identification. 
  
  This paper aims to aid researchers and analysts in automatically differentiating between collections of phishing pages. Based on our evaluation, we find that in most cases, this happens by separating the different underlying kits via the browser APIs executed. Our system has an accuracy of 93\% on a dataset of pages and kits collected from the wild. With a curated mapping of techniques to browser APIs and over 500,000 pages in which we identify \totalClusters{} clusters, we explore what techniques are universal, widespread across kits, or kit-specific.
  We find UI interactivity and basic fingerprinting (User-agent, Cookies, and referrer) to be the most universal techniques, present in 90\% and 80\% of the clusters. On the flip side, we find mouse detection (via the mouse browser API) to be among the rarest behaviors, despite one of the clusters being a deployment of a 7-year-old open-source phishing kit. 
  Prior work has noted an increased complexity of client-side JavaScript included in phishing pages; overall, our methods and findings show that browser API usage can leverage this against adversaries to differentiate phishing pages originating from different kits and better understand the breakdown of behaviors in the ecosystem. 
  \end{abstract}