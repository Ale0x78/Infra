\begin{abstract}
In 2024, the Anti-Phishing Work Group identified over one million phishing pages. Phishers achieve this scale by using phishing kits --- ready-to-deploy phishing websites --- to rapidly deploy phishing campaigns with specific data exfiltration, evasion, or mimicry techniques.
In contrast, researchers and defenders continue to fight phishing page-by-page and rely on manual analysis to recognize static features for kit identification, researchers better characterize trends in the face of mass-deployed kits, and defenders to tie kits to specific malicious actors. 
  
This paper aims to aid researchers and analysts by automatically differentiating groups of phishing pages based on the underlying kit, automating a previously manual process, and enabling us to measure how popular different client-side techniques are across these groups. For kit detection, our system has an accuracy of 95\% on a ground-truth dataset of \totalNumberOfKitFamilies{} kits families deployed across \totalKitPages{} phishing URLs. When adjusted for an over-represented kit in ground truth, we get an accuracy of 69\%, more than double that of prior attempts to tackle this problem automatically. 
On an unlabeled dataset, we leverage the complexity of \totalPagesClusterable{} phishing pages' JavaScript logic to group them into \totalClusters{} clusters, annotating the clusters with what phishing techniques they employ. We find that UI interactivity and basic fingerprinting are universal techniques, present in 90\% and 80\% of the clusters, respectively. On the other hand, mouse detection via the browser's mouse API is among the rarest behaviors, despite being used in a deployment of a 7-year-old open-source phishing kit.
We show that the sophistication of the client-side code and mass-deployment of kits can be measured by observing their browser API traces, and leverage traces obtained from \totalPagesClusterable{} to categorize different client-side techniques.
\end{abstract}