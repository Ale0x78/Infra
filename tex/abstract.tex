\begin{abstract}
In 2024, the Anti-Phishing Work Group identified over one million phishing pages. Phishers achieve this scale by using phishing kits --- ready-to-deploy phishing websites --- to rapidly deploy phishing campaigns with specific data exfiltration, evasion, or mimicry techniques.
In contrast, researchers and defenders continue to fight phishing on a page-by-page basis and rely on manual analysis to recognize static features for kit identification. 
  
This paper aims to aid researchers and analysts by automatically differentiating groups of phishing pages based on the underlying kit, automating a previously manual process, and enabling us to measure how popular different client-side techniques are across these groups. For kit detection, our system has an accuracy of \gtFMI{} on a ground-truth dataset of \totalNumberOfKitFamilies{} kits families deployed across \totalKitPages{} phishing URLs.
On an unlabeled dataset, we leverage the complexity of \totalPagesClusterable{} phishing pages' JavaScript logic to group them into \totalClusters{} clusters, annotating the clusters with what phishing techniques they employ. We find that UI interactivity and basic fingerprinting are universal techniques, present in 90\% and 80\% of the clusters, respectively. On the other hand, mouse detection via the browser's mouse API is among the rarest behaviors, despite being used in a deployment of a 7-year-old open-source phishing kit.
Our methods and findings provide new ways for researchers and analysts to tackle the volume of phishing pages.
\end{abstract}