\section{Ethical consideration}
\label{sec:ethics}
This research relied on publicly and commercially available URLs detected using proprietary methods to be phishing websites. No additional threat intelligence is attached to the URLs we are planning to release, and to ensure that any confidential information is not accidentally leaked through the URLs (like GET parameters that are indications of a test submission before the URL was submitted to the feeds), we blind the GET parameters of the URLs we visited upon release. 

We did not collect or store any identifiable information about the individuals behind phishing kits or the pages, and we did not conduct any live testing of their networks, as we only visited at most twice upon ingestion.
\section{Open Science policy}
\label{sec:osp}
We support our dataset being used for any follow-up study of phishing ecosystem or reproducibility work. Our datasets are available for vetted researchers \emph{upon request}. Releasing the phishing kits to be available to the public poses a security risk, as it would make these ready-to-deploy phishing pages available for anyone to modify and use. We also observed names, addresses, card numbers, and IP addresses in assets of phishing kits we have collected, which could effectively be used to identify prior victims of the pages. 