\section{Ethical consideration}
\label{sec:ethics}
This research relied on publicly and commercially available URLs detected using proprietary methods to be phishing websites. All of the crawling traffic originated from a network designated for web measurement, with the researchers monitoring the abuse contact for that IP space. We did not collect or store any identifiable information about the individuals behind phishing kits or the pages, and we did not conduct any live testing of their infrastructure, as we only visited at most twice upon ingestion. The phishing kits, on the other hand, contain identifiable information about their victims and Telegram API keys to access exfiltration group chats, and can be trivially manipulated to bypass current signature-based detection tools and redeployed in the wild. For this reason, we require a data-sharing agreement before we share the kits collected with any future researchers. 
\section{Open Science policy}
\label{sec:osp}
We support the use of our data and tools by other researchers for any follow-up study of the phishing ecosystem or reproducibility work. We have made all of our analyses and data collection code available on \href{https://anonymous.4open.science/r/NetGains/README.md}{anonymous.4open.science/r/NetGains} for review.
However, due to ethical concerns highlighted above, the data is available for researchers \emph{upon request}. Besides containing URLs shared through data agreements with the feed operators, collected screenshots, HAR archives, and VisibleV8 logs amounting to 6.3TB of data, which would contain identifiable information about the reviews in packed binary archives (catapult's HAR files), and create a heavy load on any data-sharing platform used.