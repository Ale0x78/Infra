\section{Conclusion}
\label{sec:conclusion}

In this paper, we provided a workflow for researchers and analysts to automatically differentiate between a collection of phishing pages, based on a common underlying kit or shared techniques, if the behaviors are too generic. We show an accuracy of \gtFMI{} against a dataset of pages and kits collected from the wild. With a curated mapping of techniques to browser APIs and \totalPagesClusterable{} pages in which we identify \totalClusters{} clusters, we explore what techniques are universal, widespread across kits, or kit-specific.  

% Phishing kits, ready-to-deploy software packages for phishing websites, have proliferated the number of web phishing attacks launched daily. Last year alone, the Anti-Phishing Work Group reported over one million phishing pages. This volume presents new challenges for security researchers and analysts, as large quantities of pages employing behaviors of interest (data exfiltration, evasion, or mimicking techniques) can originate from a singular phishing kit while being deployed across different infrastructures throughout the year. At the same time, the whole ecosystem is examined page by page, and manual analysis is required to recognize static features for kit identification. 
  
% This paper aims to aid researchers and analysts in automatically differentiating between collections of phishing pages from different underlying kits via the browser APIs executed and hierarchical clustering. We show an accuracy of 97\% against a dataset of pages and kits collected from the wild. With a curated mapping of techniques to browser APIs and over 500,000 pages in which we identify \totalClusters{} clusters, we explore what techniques are universal, widespread across kits, or kit-specific. Prior work has noted an increased complexity of client-side JavaScript included in phishing pages; overall, our methods and findings show that browser API usage can leverage this against adversaries to differentiate phishing pages originating from different kits and better understand the breakdown of behaviors in the ecosystem.