\section{Related Work (.5-1 page)} 
\label{sec:related}



%% \subsection{Phishing ecosystem}
%\myparagraph{Measurement studies of phishing sites} Recent work in analizing phishing websites from phishing feeds includes Crawlphish \cite{crawlphish}, which uses a modified webkit engine to force execute all the paths in client-side javascript to identify cloaking behavior, Rods with Laser Beams \cite{sanchez-rolaRodsLaserBeams2023} which uses a extension to capture set list of fingerprinting APIs, and identified the user of third party fingerprinting scripts in phishing pages that are not of the original target page, and Catching Phishers By Their Bait \cite{bijmansCatchingPhishersTheir2021}, which studied and identified phishing kits via manually crafted DOM and Javascript fingerprints.
%\todocite{SUNSET TO SUNRISE}

%Prior work by Lim \ et al {} has looked at client-side resources loaded by phishing pages, looking at shared scripts, CSS resources, and DOM similarities, and identifying what versions of modern libraries they use.


%\myparagraph{Phishing kit analysis} Prior work has examined phishing kits to identify tradecraft within the ecosystem \todocite{OAST}, studying the similarity of kits observed \todocite{kitphishA AND THE OTHER ONE}, or to identify pages targetting a particular theme.

%\myparagraph{Phishing page identification} Everyone under the sun has thrown ML at this problem, some state-of-the-art work has looked at constructing knowledge-based detection tools, and ofc, someone has done LLM \todocite{All of that}
%To our knowlage, we are the first paper to automatically isolate, and craft detection fingerprints, and we are first to do it at the level that VV8 lets us do it.

%% \subsection{Phishing Kits}
