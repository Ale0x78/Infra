\section{Related Work} 
\label{sec:related}

\subsection{Phishing}
\myparagraph{Phishing detection}
There is a wealth of research on phishing detection as the tug-of-war between adversaries and security professionals continues. Recently, Liu~\etal{} and Abdelnabi~\etal{} have deployed vision-based techniques to detect phishing pages ~\cite{liu_knowledge_2023,abdelnabi_visualphishnet_2020,liu_inferring_2022}. ~\cite{liu_knowledge_2023} also presents over 6,000 phishing kits analyzed as part of the work. With adversarial attacks ensuring that the page looks different to crawlers and analysis, some have turned to extracting features from the URLs themselves, more recently via LLMs in ~\cite{Chiba2024DomainLynxLL,phishReplicant} and earlier via statistical models and machine learning in ~\cite{Shirazi2018Kn0wTD,Le2018URLNetLA,Rao2018DETECTIONOP,Verma2017WhatsIA}


\myparagraph{Studying and combating adversarial techniques} Divakaran~\etal{} in ~\cite{divakaranPhishingDetectionLeveraging2022} reaffirms the need to keep up with the latest adversarial techniques to build better detection systems for phishing. Prominent work in this area includes ~\cite{zhangCrawlPhishLargescaleAnalysis2021} by Zhang~\etal{}, which uncovered and categorized many novel client-side techniques by forcing the execution of phishing pages to trigger the cloaking behavior. Acharya~\etal{} in ~\cite{acharyaPhishPrintEvadingPhishing2021} uncovered that phishing pages can successfully evade blocklists by knowing how to identify their crawlers, and Oest~\etal{} in ~\cite{oestPhishFarmScalableFramework2019} demonstrated that cloaking from non-mobile based devices as a phishing page can ensure that your page goes unmarked by the blocklists for more than 48 hours.

Kondracki~\etal{} in ~\cite{kondrackiCatchingTransparentPhish2021} uncovered a massive blindspot of the phishing detection ecosystem that was Man-in-the-middle phishing kits. Kits that would transparently forward your connection to the target page, mimicking brand logos on pages like Outlook without any configuration. Fortunately, the authors addressed the blind spot by demonstrating that these proxies remain fingerprintable using TLS fingerprinting. ~\cite{tzschoppe_browser---middle_2023} proposed a similar attack, however, one that used JavaScript and NoVNC to trick the user into signing into their account through a VNC session in their browser. 

With adversaries becoming creative with their evasions and obfuscation techniques, some novel defenses have also opted to think outside the box. Zhang~\etal{} in ~\cite{zhangImSPARTACUSNo2022} proposed a phishing defense solution that leverages the high likelihood of a phishing page cloaking away from a crawler to the defender's advantage. They demonstrated that a web browser configured to look like a crawler will trigger a cloaking response from phishing pages, ensuring that your users never see the page while maintaining compatibility with all of Alexa's Top One Million websites. Meanwhile, using CAPTCHAs ~\cite{teoh_phishdecloaker_2024} utilized vision-based models to combat phishing pages. 

To better understand why, other than cloaking, phishing pages may choose to fingerprint, Lin~\etal{} in ~\cite{linPhishSheepClothing2022} showed that browser fingerprints could be successfully used to bypass multi-factor authentication, a system meant to be a last line of defense against stolen credentials, for 10 out of 16 websites that provide popular services. 


\myparagraph{Phishing kits}
Much can be studied about the phishing ecosystem via phishing kits. Cova~\etal{} in ~\cite{freephish} uncovered that most "free" phishing kits contain a backdoor, effectively serving as a way to offload the deployment of a campaign to a 3rd party while siphoning off their stolen credentials. 

Similar to our goals, PhishKitA~\cite{castano_phikita_2023} uses a dataset of phishing kits gathered through \textbf{KitPhisher} and a collection of features extracted from the HTML DOM to classify websites into their matching kit. They achieved an F1 score 0.91 when classifying 2,000 pages (1,141 benign and 859 phishing) from features extracted from their kit. However, their multi-class classifier for identifying the kit only achieved an F1 score of 0.39. Merlo~\etal{} in ~\cite{merlo_phishing_2022} further expanded on our understanding of phishing kit lineages by looking at over 20,000 phishing kits and identifying, via token similarity, most of them as clones of one another or previously encountered kits.
Prominent work in extending our understanding of phishing attacks includes Han~\etal{} in ~\cite{hanPhishEyeLiveMonitoring2016}, where they monitor the deployment of phishing kits by adversaries that compromise vulnerable web servers by hosting a well-sandboxed honeypot. They collected 643 phishing kits and established that kits take minutes to install and test and can remain undetected for weeks. Using these kits, they were also able to identify evasion techniques used by these kits, like path randomization per visit, which back then was enough to bypass Google SafeBrowsing. 

In ~\cite{oest_inside_2018}, Oest~\etal{} manually analyzed phishing kits to establish the taxonomy for server-side cloaking, and in ~\cite{bijmans_catching_2021}, Bijmans~\etal{}, after collecting phishing kits by watching TLS transparency logs to identify Dutch brank phishing domains, manually created a fingerprint from static features to analyze their prevalence in the wild. 

More recently, Lee~\etal{} in ~\cite{lee_beneath_2024} provides a server-side script (PHP) level analysis of phishing kits, finding that dynamically generated URLs are still standard in the ecosystem and observing seasonality in the kits they were able to obtain. 

\myparagraph{Extending the understanding of the phishing ecosystem}

Similar to our methods, Rola~\etal{} in ~\cite{sanchez-rolaRodsLaserBeams2023} deployed a modified Chromium browser to gather data and analyze phishing website browser APIs utilizing a pre-selected API list focused on first and third-party scripts for phishing pages. They find that the majority of the most visited phishing pages (identified via browser telemetry data) deploy fingerprinting scripts, sometimes varying from the ones of the original brand they portray. At the same time, they accessed this at a script level, reinforcing our finding that phishing pages vary vastly from their original page.

Oest~\etal{} in ~\cite{oest_sunrise_nodate} demonstrates the full lifecycle of a phishing campaign by employing the fact that phishing pages often copy assets from the target domain and refer the victim back to the original page afterward. By collaborating with a significant financial institution, they developed a framework for leveraging this data to track a phishing page from its deployment to blocklists flagging the page as phishing. ~\cite{oest_sunrise_nodate} observed all techniques highlighted by prior work: cloaking, user-specific URL generation, man-in-the-middle proxies, and short-lived bursty attacks. Expanding our understanding of the victim experience on a phishing website, Subramani~\etal{} in ~\cite{subramani_phishinpatterns_2022} developed a crawler. 

\subsection{Dynamic analysis of webpages} Our work shares a methodology for dynamic analysis enabled by web measurement frameworks like OpenWPM~\cite{englehardt2016census} and VisibleV8~\cite{vv8-imc19}. Su~\etal{} used VisibleV8 traces and taint analysis in ~\cite{jsufp} to discover emerging fingerprinting techniques. Sarker~\etal{} used VisibleV8 to create an oracle for detecting obfuscation~\cite{jsobf-imc20}. Such an oracle was made possible by the observation that VisibleV8 marks the execution of an API at a given source line. At the same time, obfuscation techniques ensure that the API is not textually available there. And Pantelaios~\etal{} used a combination of VisibleV8 and force execution modifications to the Chromium engine to identify and defeat JavaScript evasion techniques while also leveraging API traces and clustering to identify previously unlabeled malicious extensions~\cite{fv8-sec24}. Iqbal~\etal{} used OpenWPM to capture execution traces from tranco top 100K URLs of Tranco, training a classifier on a mixture of dynamic and static features extracted from the JavaScript's AST and execution traces, respectively, to achieve a 99.8\% accuracy in identifying fingerprinting scripts online. 

