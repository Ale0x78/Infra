\section{Data characterization} 
\label{sec:data}
This paper utelizes two distinct dataset.

\subsection{Phishing Kits and Ground truth}
The ground truth data for this paper, is a mapping of \totalNumberOfKits{} to \totalKitPages{} phishing pages. We collected \totalZipFiles{} archive files from these feeds in total. Only two were encrypted, which we treated as unique kits based on their SHA256 hash. We further filtered the archive files by ensuring that they container at least a single code file as identified by Magika. This further reduced our archive count to \totalNumberOfKits{}.
Finally, we group these kits into \totalNumberOfKitFamilies{}, \numReviewers{} manually reviewed \pairsReviews{} pairs of kits which had a file simularity between 10-80\%. 

Inline with prior work, XXX\% of these phishing kits were written in php, however we do note that XXX\% phishing kits shipped with HTML/CSS/JS, and we found 2 that were written in python. During our manual analisys of these kits, we concured with prior work that many of these kits re-used eachother's code, especially when it came to server-side IP blacklists. We found two module-like anti-bot detection files frequiently redistributed across kits. 
\alex{This section should talk about HOW long do we see these kits for on average. We are the first study since Oest to look at kits for over a year in these feeds}
\subsection{Unlabled Phishing pages}
In total, we crawled for \daysCrawled{} days collecting browser traces from \totalPageWithJavascript{} pages, out of which only \totalPagesWithJavascriptFP{} qualified for clustering by executing at least 8 JavaScript APIs in a first-party context. \textbf{376,762} did not execute any \js{} in a first party context and the rest \textbf{418,641} did not execute at least 8 distinct browser APIs. Furthermore, 99,464 pages were clustered as noise by HDBSCAN or formed a cluster where all the pages had no common APIs. As shown in Figure~\ref{fig:pages_clusters_per_month}, in contrast to the number of pages we observe monthly, we can substantially reduce the scope of the phishing ecosystem based on common behaviors. On average, we reduce the monthly traffic to 6\% ($\sigma=2$) of the traffic of urls. Before the 2nd DBSCAN merge of the 26,936 local clusters, Figure~\ref{fig:localClusters} shows the distribution of the silhouette score of the local clusters. With an average score of 0.8, these clusters are incredibly well formed and, on average, pages inside a cluster contain 64 APIs in common. Once clustered together, the silhouette score of the final clusters was \textbf{0.6}. The average cluster in our dataset contained 49 pages ($\sigma=468$) and existed for 41 days ($\sigma=77$).

We collected 2,590 files with \kp{} for the ground truth dataset, and de-duplicated them into 2,516 phishing kits. Only \textbf{519} have at least two URLs that we crawled, for a total of 4,438 pages. When evaluating the ground truth, we dropped the API requirement to four in a first-party context.